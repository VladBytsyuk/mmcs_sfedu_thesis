% В этом файле следует писать текст работы, разбивая его на
% разделы (section), подразделы (subsection) и, если нужно,
% главы (chapter).

% Предварительно следует указать необходимую информацию
% в файле SETUP.tex

\input{preamble.tex}

\begin{document}

\Intro

Здесь нужно написать введение.

% Если typeOfWork в SETUP.tex задан как 2 или 3, то начинать
% надо не с section (раздел), а с главы (chapter)
\section{Знакомство}
\subsection{Erlang}
Erlang - функциональный язык программирования, созданный для разработки
распределенных динамических систем. Основные его приемущества: быстрая и 
эффективная разработка, устойчивость системы к аппаратным сбоям и 
возможность обновления всей системы без остановки программ.

\subsubsection{Переменные и атомы} 
Переменные в Erlang объявляются следующим образом:
\begin{lstlisting}
X = 42.
\end{lstlisting}
Все переменные начинаются с заглавной буквы. В Erlang переменным 
можно присваивать значения только один раз. Переменная которой 
значение уже присвоено называется связанной. В противном случае 
она называется свободной. Попытка присвоить связанной переменной 
новое значение приведет к сообщению об ошибке.

Атомы используются для представления нечисловых констант. 
\begin{lstlisting}
monday.
\end{lstlisting}
Все атомы начинаются с прописной буквы. Также атомы могут быть 
заключены в одиночные кавычки ('). 
\begin{lstlisting}
'January'.
\end{lstlisting}
В таком случае атом может начинаться с большой буквы.
Значением атома является сам атом.  


\subsubsection{Сопоставление по образцу}
В Erlang символ = означает операцию сопоставления по
образцу.
\begin{lstlisting}
2 + 4 = 3 + 3.
\end{lstlisting}
В процессе выполнения данного участка кода сначала вычислится 
3 + 3, далее вычислится 2 + 4, а потом сопоставятся 2 результата.
\begin{lstlisting}
Y = 6 * 7.
\end{lstlisting}
В процессе выполнения данного участка кода сначала вычислится
6 * 7, а потом так как переменная Y свободная, то ее значение станет 
равно значению правой стороны выражения, и равенство станет верным.


\subsubsection{Кортежи}
Кортеж - единая группа из фиксированного числа объектов. Группа является 
анонимной, как и каждое отдельное поле кортежа.
\begin{lstlisting}
{1, september, 2012}.
{point, 6, 7}.
\end{lstlisting}
Часто первым элементом кортежа используют атом, котрый описывает этот кортеж.

Кортежи могут быть вложенными друг в друга.
\begin{lstlisting}
{date,
	{day, 1},
	{month, september},
	{year, 2012}
}.
\end{lstlisting}

Возможно присваивать переменным значения отдельных элементов кортежа.
\begin{lstlisting}
{Day, Month, Year} = {1, september, 2012}.
\end{lstlisting}
В переменную Day запишется значение 1, в Month - september, а в Year - 2012.

\begin{lstlisting}
{Name, _} = {joe, armstrong}.
\end{lstlisting}
Символ \_ называется анонимной переменной. Такой переменной не привыязывается 
соответствующее значение. Результатом выполнения данного участка кода Это привязка
переменной Name значения joe.


\subsubsection{Списки} 
Списки используются для хранения различых данных.
\begin{lstlisting}
[{joe, armstrong}, {1, september, 2012}, 42].
\end{lstlisting}
Головой списка называется его первый элемент. Если удалить голову из списка,
то останется хвост исходного списка. 
\begin{lstlisting}
[H|T] = [{joe, armstrong}, {1, september, 2012}, 42].
\end{lstlisting}
В результате к переменной H будет привязано значение
\begin{lstlisting} 
{joe, armstrong}
\end{lstlisting}
а переменной T значение 
\begin{lstlisting}
[{1, september, 2012}, 42].
\end{lstlisting}

Следующим образом можно добавлять элементы в список:
\begin{lstlisting}
[{82, 56}, morning|T].
\end{lstlisting}
Результатом будет список
\begin{lstlisting} 
[{82, 56}, morning, {1, september, 2012}, 42].
\end{lstlisting}

Конкатенация списков производится следующим образом:
\begin{lstlisting}
[34, red] ++ [{point, 6, 7}].
\end{lstlisting}
Результатом будет список 
\begin{lstlisting}
[34, red, {point, 6, 7}].
\end{lstlisting}


\subsubsection{Функции}
Рассмотрим описание функций в Erlang на примере нахождения площади прямоугольника и круга.
\begin{lstlisting}
area({rectangle, Width, Height}) -> Width * Height;
area({circle, Radius}) -> 3.14159 * Radius * Radius.
\end{lstlisting}
Функция area содержит 2 варианта сопоставления аргументов - клаузы. 
Первый вариант необходима для находения площади прямоугольника, а
второй - круга. 
Результатом вызова 
\begin{lstlisting}
area({rectangle, 2, 3}).
\end{lstlisting}
будет число 6. Выберется первый вариант выполнения функции, так как первым элементом 
кортежа является rectangle.



\subsection{Красно-черные деревья}
Красно-черное дерево - двоичное дерево поиска, узлы которого 
разделены на красные (red) и черные (black). Для таких деревьев
должны выполняться красно-черные свойства (RB properties), 
гарантирующие, что глубины любых двух листьев отличаются не более
чем в 2 раза.

Узлы красно-черного дерева обычно содержат следующие поля:
\begin{enumerate}
	\item Значение
	\item Цвет
	\item Родитель
	\item Левый ребенок
	\item Правый ребенок
\end{enumerate}	

Важно отметить, что если ребенок или родитель отсутствует, то
соответсвующее поле содержит черный лист.

Рассмотрим упомянутые выше красно-черные свойства (RB properties):
\begin{enumerate}
	\item Каждый узел дерева - либо красный, либо черный.
	\item Корень дерева - черный.
	\item Каждый лист - черный.
	\item Если узел красный, то оба его ребенка черные.
	\item Все простые пути, идущие от корня к листьям, содержат 
		  одинаковое количество черных листьев.
\end{enumerate}

Для удобства работы, все листья заменяются одним черным листом.
Это обычный узел дерева со значением nil, черным цветом и произвольными данными
о потомках. Использование подобного узла позволяет рассматривать дочерний 
по отношению к узлу черный лист как обычный узел с известным предком.

\underline{Черная высота узла X} - количество черных узлов на любом простом 
пути от узла X (не считая сам узел) к листу. Обозначим черную высоту,
как bh(X).

В соответсвии со свойством 5 - черная высота узла - точно определяемое значение,
поскольку все нисходящие простые пути из узла содержат олно и то же 
количество черных узлов.

\underline{Черная высота дерева} - черная высота его корня.

\underline{Лемма}

Красно-черное дерево с $n$ внутренними узлами имеет высоту, не превышающую 
$2\lg(n+1)$.

Операции поиска, минимума, макисмума, предков, потомков, вставки, удаления выполняется 
за время $O(\lg h)$, где $h$ - высота красно-черного дерева.

Так как операции вставки и удаления изменяют красно-черное дерево,
то в результате их работы могут нарушаться красно-черные свойства. 
Для восстановления красно-черных свойств необходимо изменить:
\begin{enumerate}
	\item Цвета некоторых узлов дерева.
	\item Родительски-дочерние связи некоторых узлов дерева.
\end{enumerate}

Последнее выполняется с помощью поворотов. Это локальные операции в
дереве поиска, сохраняющие красно-черные свойства.
Существует 2 типа поворотов: левый и правый.

\begin{figure}[H]
\centering
%Здесь могла быть ваша лягушка.
\includegraphics[width=\textwidth]{img/tan-aus.png}
\caption{\label{fig:tan-aus}Пример левого и правого поворотов.}
\end{figure}

\underline{Замечание}

	При выполнении левого поворота в узле X предполагается, что
	его правый ребенок Y не является черным узлом.
	
	При выполнении правого поворота в узле Y предполагается, что
	его левый ребенок X не является черным узлом.

Рассмотрим алгоритм вставки в красно-черное дерево. Вставка выполняется в 2 этапа:
\begin{enumerate}
	\item Вставка нового узла в красно-черное дерево, как в обычное бинарное
		  дерево поиска, и окрашивает его в красный цвет.
	\item Выполнение необходимых поворотов и перекрашиваний узлов 
		  красно-черного дерева.
\end{enumerate}

При этом возникает следующая проблема - нарушаются красно-черные свойства.
При выполнении необходимых поворотов и перекрашиваний узлов красно-черного 
дерева корень дерева может быть окрашен в красный цвет, что будет протворечить
свойству 2, а при вставки нового узла в красно-черное дерево, и окрашивании
его в красный цвет может возникнуть ситуация, когда у красного узла будет 
красный ребенок.

При вставке возможны 4 случая нарушения четвертого красно-черного свойства:
\begin{figure}[H]
\centering
\includegraphics[width=\textwidth]{img/tan-aus.png}
\caption{Пример возможных нарушений красно-черных свойств после вставки.}
\end{figure}  



\section{Несколько примеров в~\LaTeX{}}
\label{sec:examples}

Некоторые часто используемые
команды приведены в качестве примера ниже (и варианты — в
комментариях). Мы рекомендуем внимательно прочесть данный
текст и изучить его исходный код прежде, чем начинать писать
свой собственный. Кроме того, можно дать и такой совет: идущий
ниже текст не убирать до самого конца, а просто оставлять его
позади своего собственного текста, чтобы в любой момент можно
было проконсультироваться с данными примерами.

\subsection{Как вставлять листинги и рисунки}

Для крупных листингов есть два способа. Первый красивый, но в нём не допускается
кириллица (у вас может встречаться в комментариях и
печатаемых сообщениях), он представлен на листинге~\ref{list:hwbeauty}.
\begin{ListingEnv}[H]% буква H означает Here, ставим здесь,
% элементы, которые нежелательно разрывать обычно не ставят
% посреди страницы: вместо H используется t (top, сверху страницы),
% или b (bottom) или p (page, на отдельной странице)
\begin{lstlisting}
#include <iostream>
using namespace std;

int main()
{
    cout << "Hello, world" << endl;
    system("pause");
    return 0;
}
\end{lstlisting}
%следующую команду для генерации подписи можно опустить,
% хотя рекомендуется все специальные элементы (таблицы, рисунки,
% листинги) подписывать. Если подпись пропустить, листинг также не получит
% номера и на него не сошлёшься в будущем
\caption{Программа “Hello, world” на \protect\cpp}
% далее метка для ссылки:
\label{list:hwbeauty}
\end{ListingEnv}

Второй не такой красивый, но без ограничений (см.~листинг~\ref{list:hwplain}).
\begin{ListingEnv}[H]
\begin{Verb}

#include <iostream>
using namespace std;

int main()
{
    cout << "Привет, мир" << endl;
}
\end{Verb}
\caption{Программа “Hello, world” без подсветки}
\label{list:hwplain}
\end{ListingEnv}

Можно использовать первый для вставки небольших фрагментов
внутри текста, а второй для вставки полного
кода в приложении, если таковое имеется.

Если нужно вставить совсем короткий пример кода (одна или две строки), то выделение  линейками и нумерация может смотреться чересчур громоздко. В таких случаях можно использовать окружения \texttt{lstlisting} или \texttt{Verb} без \texttt{ListingEnv}. Приведём такой пример с указанием языка программирования, отличного от заданного по умолчанию:
\begin{lstlisting}[language=Haskell]
fibs = 0 : 1 : zipWith (+) fibs (tail fibs)
\end{lstlisting}
Такое решение~--- со вставкой нумерованных листингов покрупнее
и вставок без выделения для маленьких фрагментов~--- выбрано,
например, в книге Эндрю Таненбаума и Тодда Остина по архитектуре
компьютера~\autocite{TanAus2013} (см.~рис.~\ref{fig:tan-aus}).

Наконец, для оформления идентификаторов внутри строк
(функция \lstinline{main} и тому подобное) используется
\texttt{lstinline} или, самое простое, моноширинный текст
(\texttt{\textbackslash texttt}).

\begin{figure}[p]% p означает, что нужно выделить для рисунка
% отдельную страницу; применяется для больших рисунков
\centering
%Здесь могла быть ваша лягушка.
\includegraphics[width=\textwidth]{img/tan-aus.png}
\caption{\label{fig:tan-aus}Пример оформления листингов в~\autocite{TanAus2013}}
\end{figure}

Использовать внешние файлы (например, рисунки) можно и на \href{http://overleaf.com}{overleaf.com}: ищите кнопочку upload.

\subsection{Как оформить таблицу}

Для таблиц обычно используются окружения table и tabular --- см. таблицу~\ref{tab:widgets}. Внутри окружения tabular используются специальные команды пакета booktabs — они очень красивые; самое главное: использование вертикальных линеек считается моветоном.

\begin{table}
\centering
\caption{\label{tab:widgets}Подпись к таблице --- сверху}
\begin{tabular}{llr}
\toprule
\multicolumn{2}{c}{Item} \\
\cmidrule(r){1-2}
Животное  & Описание    & Цена (\$) \\
\midrule
Gnat      & per gram    & 13.65      \\
          & each        & 0.01       \\
Gnu       & stuffed     & 92.50      \\
Emu       & stuffed     & 33.33      \\
Armadillo & frozen      & 8.99       \\
\bottomrule
\end{tabular}
\end{table}

\subsection{Как набирать формулы}

\LaTeX{} is great at typesetting mathematics. Let $X_1, X_2, \ldots, X_n$ be a sequence of independent and identically distributed random variables with $\text{E}[X_i] = \mu$ and $\text{Var}[X_i] = \sigma^2 < \infty$, and let
$$S_n = \frac{X_1 + X_2 + \cdots + X_n}{n}
      = \frac{1}{n}\sum_{i}^{n} X_i$$
denote their mean. Then as $n$ approaches infinity, the random variables $\sqrt{n}(S_n - \mu)$ converge in distribution to a normal $\mathcal{N}(0, \sigma^2)$.

\subsection{Как оформлять списки}

Нумерованные списки (окружение enumerate, команды item)…

\begin{enumerate}
  \item Like this,
  \item and like this.
\end{enumerate}

\dots маркированные списки \dots

\begin{itemize}
  \item Like this,
  \item and like this.
\end{itemize}

\dots списки-описания \dots

\begin{description}
  \item[Word] Definition
  \item[Concept] Explanation
  \item[Idea] Text
\end{description}

\Conc

Помните, что на все пункты списка литературы должны быть ссылки. \LaTeX\ просто не добавит информацию об издании из bib"/файла, если на это издание нет ссылки в тексте. Часто студенты используют в работе  электронные ресурсы: в этом нет ничего зазорного при одном условии: при каждом заимствовании следует ставить соответствующую ссылку. В качестве примера приведём ссылку на сайт нашего института~\autocite{mmcs}.

Для дальнейшего изучения \LaTeX\ рекомендуем книгу Львовского~\autocite{Lvo2003}: она хорошо написана, хотя и несколько устарела.
Обычно стоит искать подсказки на
\href{http://tex.stackexchange.com/}{tex.stackexchange.com}, а также
читать документацию по установленным пакетам с помощью
команды
\begin{Verb}
texdoc имя_пакета
\end{Verb}
или на \href{http://ctan.org/}{ctan.org}.

% Печать списка литературы (библиографии)
\printbibliography[%{}
    heading=bibintoc%
    %,title=Библиография % если хочется это слово
]
% Файл со списком литературы: biblio.bib
% Подробно по оформлению библиографии:
% см. документацию к пакету biblatex-gost
% http://ctan.mirrorcatalogs.com/macros/latex/exptl/biblatex-contrib/biblatex-gost/doc/biblatex-gost.pdf
% и огромное количество примеров там же:
% http://mirror.macomnet.net/pub/CTAN/macros/latex/contrib/biblatex-contrib/biblatex-gost/doc/biblatex-gost-examples.pdf

\end{document}
